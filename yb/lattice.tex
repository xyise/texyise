\documentclass{texyise}

\documentTitle{On lattice-based pricing}
\documentAuthor{...}
\documentDate{...}
\documentClassification{...}

\usepackage{amsmath}
\usepackage{amsfonts}
\usepackage{bm}
\usepackage{hyperref}

\newcommand{\indexset}{\mathcal{I}}
\newcommand{\valuebackward}{\mathcal{B}}
\newcommand{\valuebackwardnodisc}{\mathcal{L}}
\newcommand{\valuearbfree}{\mathcal{A}}

\begin{document}

\section{Introduction}

\subsection{2-factor skew model}

In the Yield Book, the 2-factor skew model for short rate $r(t)$ is given as
\begin{equation}
    r(t) = f(t) e^{x(t)} - \beta  \label{E:2fskew}
\end{equation}
where $\beta$ is a constant and $f(t)$ is a determinstic function in time $t$. The state variable $x(t)$ is driven by a two-factor process with the following dynamics:
\begin{eqnarray}
    dx(t) &=& (-a(t)x(t)+u(t))dt + \sigma_1(t) dW_1(t) \label{E:2fskew1} \\
    du(t) &=& -k(t)u(t)dt + \sigma_2(t) dW_2(t) \label{E:2fskew2}
\end{eqnarray}
where $W_1(t)$ and $W_2(t)$ are independent Brownian motions and the parameters $a(t)$, $k(t)$, $\sigma_1(t)$, $\sigma_2(t)$ are deterministic functions in time $t$.
Denoting the valuation time by $t_0$, the initial values of the state variables are given as
\begin{equation}
    x(t_0) = 0 \quad \text{and} \quad u(t_0) = 0.
\end{equation}

\subsection{Purpose of This Document}

Although often termed as a `tree' method, the Yield Book utilises a lattice-based pricing approach to implement the 2-factor skew model for pricing interest rate derivatives. While it shares a lot of similarities with the tree method, the lattice-based pricing approach has distinct features. The purpose of this note is to provide a more detailed documentation of the lattice-based pricing approach, supplementing the existing documentation in \cite{2fs-citi-mv}.

Since the lattice-based pricing approach is a general framework not limited to the 2-factor skew model, we consider the following more general form of a term-structure model: The short rate $r(t)$ is given as
\begin{equation}
    r(t) = R(\bm{y}(t), t) \label{E:short-rate-generic}
\end{equation}
where the 2-dimentional state vector $\bm{y} = (y_1, y_2)$ is driven by the following process:
\begin{equation}
    d\bm{y}(t) = \bm{\mu}(\bm{y}(t), t) dt + \bm{\sigma}(\bm{y}(t), t) d\bm{W}(t) 
    \quad\text{with}\quad \bm{y}(t_0) = \bm{0}.
    \label{E:state-variable-generic}
\end{equation}
Here, $R$ is a deterministic scalar function, $\bm{\mu}$ is a vector function, $\bm{\sigma}$ is a diagonal matrix function, and $\bm{W}(t)$ is a vector of independent Brownian motions.

\textbf{Note}: While it is straightforward to extend the model with more than two factors, the computational complexity of applying the lattice-based approach increases significantly with the number of factors. In practice, the approach is typically limited to models with at most two factors.

\section{Lattice-Based Pricing Approach}
\label{S:lattice-based-pricing}

\textbf{Note}: As we aim to be as precise as possible in describing the lattice-based pricing approach, the notation used in this section might appear quite complex. 


\subsection{Lattice Structure and Pricing}
\label{S:lattice-structure}

The structure of the lattice-based approach consistes of a collection of nodes over a time grid and transition probabilities between the nodes:
\begin{itemize}
    \item time grid:
    \begin{equation}
        t_0 < t_1 < \cdots < t_N \label{E:time-grid}
    \end{equation}
    where $t_0$ is the valuation time and $t_N$ is the final date pertinent to the pricing of the derivative contracts under considerations (e.g.\ the maturity in case of pricing a bond). 
    \item nodes: For each time $t_n$ in the time grid, we have a set of discrete state vectors, denote by $\mathcal{Y}_n$, as a 2-dimensional Cartesian product:
    \begin{equation}
        \mathcal{Y}_n := Y_{1,n} \times Y_{2,n} = \left\{ y_{1,n}^{m_1} \right\}_{m=1}^{M_1} \times \left\{ y_{2,n}^{m_2} \right\}_{m=1}^{M_2}
        \label{E:node-set}
    \end{equation}
    A {\em node} belonging to this set can be written in a vector form:
    \begin{equation}
        \bm{y}_{n}^{\bm{m}} := (y_{1,n}^{m_1}, y_{2,n}^{m_2}) \quad\text{for}\quad m_1 = 1, \ldots, M_{n,1} \quad\text{and}\quad m_2 = 1, \ldots, M_{n,2}.
    \end{equation}
    where the index vector $\bm{m} = (m_1, m_2)$ runs over $\indexset_n$, which denotes the Cartesian product of the indices $m_1$ and $m_2$.

    \item transition probabilities: For each time $t_n$, we have a transition probability matrix $\bm{Q}_n$ that describes the transition of the state vector from time $t_n$ to time $t_{n+1}$:
    \begin{equation}
        \bm{Q}_n = \left[ q_{n}^{\bm{s},\bm{e}} \right]_{\bm{s} \in \indexset_n, \bm{e} \in \indexset_{n+1}}.
    \end{equation} 

\end{itemize}

Once the lattice structure is constructed, the pricing of a derivative contract is carried out by the standard backward induction method. Specifically, the price of the derivative contract at time $t_n$ is given by the following recursive formula:
\begin{equation}
    V(\bm{y}_{n}^{\bm{s}}, t_n) = D_n^{\bm{s}}\sum_{\bm{e} \in \indexset_{n+1}} q_{n}^{\bm{s},\bm{e}} \times V(\bm{y}_{n+1}^{\bm{e}}, t_{n+1}) \label{E:backward-induction}
\end{equation}
where $D_n^{\bm{s}}$ is the discount factor form time $t_{n+1}$ seen at time $t_{n}$ for the state vector $\bm{y}_n^{\bm{s}}$, and given by
\begin{equation}
    D_n^{\bm{s}} = \exp\left(-R(\bm{y}_n^{\bm{s}}) (t_{n+1} -t_n) \right)
    \label{E:discount-factor-node}
\end{equation}
through the short rate model \eqref{E:short-rate-generic}. Here, the short rate is assumed constant over the time interval $[t_n, t_{n+1})$ conditional on the state vector $\bm{y}_n^{\bm{s}}$ at $t_{n}$.

Applying the backward induction \eqref{E:backward-induction} to a given payoff function $g(\bm{y})$ at time $t_n$, the value of the derivative contract at the valuation time $t_0$ is denoted by $\valuebackward[g, t_n]$. Then, the lattice structure {\em would} be set up such that it converges to the true value of the derivative contract through the arbitrage-free pricing principle:
\begin{equation}
    \mathbb{E}_{\bm{y} \sim \bm{y}(t_n)}\left[ e^{-\int_{t_0}^{t_n} r(s)ds} g(\bm{y}) \right] \approx \valuebackward[g,t_n] \label{E:ideal-lattice}
\end{equation}
But, as we will see in the next section, the lattice structure is constructed such that \eqref{E:ideal-lattice} holds under the assumption of zero rate ($r(t) = 0$). This approach makes the model calibration process more modular, allowing it to be executed in two stages:
\begin{itemize}
    \item constructing the lattice structure based on the dynamic model \eqref{E:state-variable-generic}
    \item fitting the short rate model \eqref{E:short-rate-generic} to the given term-structure
\end{itemize}

\subsection{Lattice structure construction}
\label{S:lattice-structure-construction}

In this section, we describe the procedure of constructing the lattice structure, given a set of the model parameters $\bm{\mu}$ and $\bm{\sigma}$ for the state variable process in \eqref{E:state-variable-generic}.

\paragraph*{time grid:}

First, divide the time horizon into small time steps to set up the time grid \eqref{E:time-grid} as described in \S~\ref{S:time-grid}. Then, all model variables are assumed to be piecewise constant over each time intervals $[t_n, t_{n+1})$.
In particular, the state vector $\bm{y}(t_{n+1})$ conditional on $\bm{y}(t_n) = \bm{y}$ follows a multivariate normal distribution of independent variables and satisfies
\begin{equation}
    y_i(t_{n+1}) | \bm{y}(t_n) = \bm{y}  \sim \mathcal{N}(\mu_{i,n}(\bm{y},t_n), \sigma_{i,n}(\bm{y},t_n)^2)
    \label{E:conditional-distribution}
\end{equation}

\paragraph*{recursion setup:}

The nodes and transition probabilities are constructed recursively. Suppose that the nodes and transition probabilities are already constructed up to time $t_n$ in such a way that 
\begin{equation}
    \valuebackwardnodisc[g, t_n] \approx \mathbb{E}_{\bm{y}\sim\bm{y}(t_n)}\left[ g(\bm{y}) \right] 
    \label{E:lattice-calib}
\end{equation}
where $\valuebackwardnodisc[g, t_n]$ is obtained by the same backward induction \eqref{E:backward-induction}, but with the discount factor set to one. 

Since the backward induction \eqref{E:backward-induction} is a linear operation, 
$\valuebackwardnodisc[g, t_n]$ can be written as a linear combination of the values of $g(\bm{y})$ at the nodes corresponding to time $t_n$. Combined with \eqref{E:lattice-calib}, this implies that
\begin{equation}
\sum_{\bm{m} \in \indexset_n} A_{n}^{\bm{m}} \times g(\bm{y}_n^{\bm{m}}) \approx \mathbb{E}_{\bm{y}\sim \bm{y}(t_n)}\left[ g(\bm{y}) \right] 
\label{E:lattice-calib-arrow-debreu}
\end{equation}
for some coefficients $A_{n}^{\bm{m}}$.\footnote{$A_{n}^{\bm{m}}$ can be interpreted as Arrow-Debreu prices under the zero-rate assumption. In \cite{2fs-citi-mv}, the representation \eqref{E:lattice-calib-arrow-debreu}
is the starting point of the lattice-based pricing approach.}

By definition, we have
\begin{equation}
    A_{n}^{\bm{m}} = \valuebackwardnodisc[F_n^{\bm{m}}, t_n] \label{E:arrow-debreu-def}
\end{equation}
where $F_n^{\bm{m}}(\bm{y})$ is a continous approximation to the `delta' payoff function, and its value at the node points are given by
\begin{equation}
    F^{\bm{m}}(\bm{y}_{n}^{\bm{k}}) = 
    \left\{
    \begin{array}{cl}
        1 & \text{if}\quad \bm{k} = \bm{m} \\
        0 & \text{otherwise}
    \end{array}
    \right.
    \label{E:delta-payoff}
\end{equation}
Specifically, $F_n^{\bm{m}}(\bm{y})$ is defined as
\begin{equation}
    F_n^{\bm{m}}(\bm{y}) = F_{1,n}^{m_1}(y_1) \times F_{2,n}^{m_2}(y_2)
\end{equation}
where $F_{i,n}^{m_i}(y_i)$ is the natural cubic spline function that connects the above payoff values at the nodes for the $i$-th state variable $y_i$ at time $t_n$. For further details, see \S~\ref{S:cubic-spline}.

Now, we proceed to construct the lattice structure for time $t_{n+1}$. 

\paragraph*{nodes:}

The nodes \eqref{E:node-set} at time $t_{n+1}$ are located by identifying an intervals that encompasses the majority of the probability distribution of each state variable $y_i(t_{n+1})$. Specifically, the interval for state variable $y_i$ is set to that that covers $\nu$ standard deviations of $y_i(t_{n+1})$ from the mean of $y_i(t_{n+1})$, and the $i$-th node set $Y_i$ in \eqref{E:node-set} is set to the $M_i$ equally spaced grid points in the interval for $y_i(t_{n+1})$.

\begin{itemize}
    \item To calculate the mean, consider a special payoff function:
    \begin{equation}
    g_{i,n}(\bm{y}) := \mathbb{E}[y_i(t_{n+1}) | \bm{y}(t_n) = \bm{y}]
    \end{equation}
    which is readily available from \eqref{E:conditional-distribution}.
    Then, we have
    \begin{eqnarray}
        \mathbb{E}[y_i(t_{n+1})] & = & \mathbb{E}_{\bm{y}\sim\bm{y}(t_n)}[\mathbb{E}[y_i(t_{n+1}) | \bm{y}(t_n) = \bm{y}]] \label{E:mean-tower}\\
                  & = & \mathbb{E}_{\bm{y}\sim\bm{y}(t_n)}[g_{i,n}(\bm{y})] \label{E:mean-by-def}\\
                  & \approx & \sum_{\bm{m} \in \indexset_n} A_{k, \bm{m}} \times g_{i,n}(\bm{y}_{n}^{\bm{m}}) \label{E:mean-app-arrow-debreu} 
    \end{eqnarray}
    where \eqref{E:mean-tower} is from the tower property of conditional expectation, \eqref{E:mean-by-def} is by the definition of the payoff function $g_{i,n}$, and \eqref{E:mean-app-arrow-debreu} is from \eqref{E:lattice-calib-arrow-debreu}.
    \item To calculate the standard deviation, consider another special payoff function:
    \begin{equation}
        h_{i,n}(\bm{y}) := \mathbb{E}[y_i(t_{n+1})^2|\bm{y}(t_n)= \bm{y}] 
    \end{equation}
    which can be calculated from \eqref{E:conditional-distribution}. Then, we have
    \begin{eqnarray}
        \mathbb{E}[y_i(t_{n+1})^2] & = & \mathbb{E}_{\bm{y}\sim\bm{y}(t_n)}[h_{i,n}(\bm{y})] \label{E:variance-by-def}\\
                  & \approx & \sum_{\bm{m} \in \indexset_n} A_{k, \bm{m}} \times h_{i,n}(\bm{y}_{n}^{\bm{m}}) \label{E:variance-app-arrow-debreu}
    \end{eqnarray} 
    which follows the same reasoning as \eqref{E:mean-app-arrow-debreu}.

    Then, the standard deviation of $y_i(t_{n+1})$ is given by the square root of the difference between the expectation of the square and the square of the expectation (a textbook formula). 
\end{itemize}

For how $\nu$ and $M_i$'s are set, see \S~\ref{S:node-grid}.

\paragraph*{transition probabilities:}

The goal is to set the transition probability matrix $\bm{Q}_n$ such that \eqref{E:lattice-calib} holds for the next time step $t_{n+1}$. To this end, consider the delta payoff function $F_{n+1}^{\bm{e}}(\bm{y})$ at time $t_{n+1}$ as defined in \eqref{E:delta-payoff}. 

On the one hand, we have
\begin{eqnarray}
    \valuebackwardnodisc[F_{n+1}^{\bm{e}}, t_{n+1}] & = & \sum_{\bm{s} \in \indexset_{n}} q_{n}^{\bm{s},\bm{e}} \times \valuebackwardnodisc[F_{n}^{\bm{s}}, t_{n}]. \label{E:transition-prob-calib}
\end{eqnarray}

On the other hand, let's define 
\begin{equation}
    f_{n}^{\bm{e}}(\bm{s}) := \mathbb{E}[F_{n+1}^{\bm{e}}(\bm{y}_{n+1})|\bm{y}(t_n)=\bm{y}_n^{\bm{s}}]
    \quad \text{for}\quad \bm{s} \in \indexset_{n} \quad\text{and}\quad \bm{e} \in \indexset_{n+1}.
\end{equation}
which can be computed numerically as shown later in this section.
Then, we have
\begin{eqnarray}
    \mathbb{E}_{\bm{y}\sim\bm{y}(t_{n+1})}\left[ F_{n+1}^{\bm{e}}(\bm{y}) \right] & = & 
    \mathbb{E}_{\bm{y}\sim\bm{y}(t_{n})}\left[ f_{n}^{\bm{e}}(\bm{y}) \right] \\
    & \approx & \sum_{\bm{s} \in \indexset_{n}} A_{n}^{\bm{s}} \times f_{n+1}^{\bm{e}}(\bm{s})
    \label{E:transition-prob-calib-2}
\end{eqnarray}
where \eqref{E:lattice-calib-arrow-debreu} is applied.
Comparing \eqref{E:transition-prob-calib} to \eqref{E:transition-prob-calib-2} alongside \eqref{E:arrow-debreu-def}, we deduce that
\begin{equation}
    q_{n}^{\bm{s},\bm{e}} = f_{n+1}^{\bm{e}}(\bm{s}).
\end{equation}

It remains to describe how $f_{n+1}^{\bm{e}}(\bm{s})$ is computed: 
\begin{eqnarray}
f_{n+1}^{\bm{e}}(\bm{s}) & := & \mathbb{E} [F_{1,n+1}^{e_1}(y_1(t_{n+1})) \times F_{2,n+1}^{e_2}(y_2(t_{n+1}))| \bm{y}(t_n) = \bm{y}_n^{\bm{s}}] \label{E:green-1}\\
    & = & \prod_{i=1,2} \mathbb{E} [F_{i,n+1}^{e_i}(y_i(t_{n+1})) | \bm{y}(t_n) = \bm{y}_n^{\bm{s}}] \label{E:green-2}
\end{eqnarray}
because the state variables $y_1(t_{n+1})$ and $y_2(t_{n+1})$ are independent conditional on the previous time step as the Brownian motions in \eqref{E:state-variable-generic} are assumed independent.
The $i$-th term in \eqref{E:green-2}, denoted by $f_{i,n+1}^{e_i}(\bm{s})$, is computed numerically using the standard Gaussian quadratures.

The independence assummption significantly reduces the computational complexity and memory requirement for the transition probability matrix. Specifically, we only need to compute and store the following number of expected values
\begin{eqnarray}
    (M_{n,1} \times M_{n,2}) \times (M_{n+1,1} + M_{n+1,2}),
\end{eqnarray}
which is considerably less than the actual number of elements in the transition probability matrix $\bm{Q}_n$
\begin{eqnarray}
    (M_{n,1} \times M_{n,2}) \times (M_{n+1,1} \times M_{n+1,2}).
\end{eqnarray}

\paragraph*{construction outputs:}

The construction outputs are the nodes and transition probabilities for each time step $t_n$ in the time grid \eqref{E:time-grid}. For transition probabilities, we store the expected values $f_{i,n+1}^{e_i}(\bm{s})$ for $i = 1, 2$, $\bm{s} \in \indexset_n$, and $e_i = 1, \cdots, M_{i,n+1}$.

\subsection{Fitting to term-structure}
\label{S:fitting-term-structure}

The short rate model \eqref{E:short-rate-generic} is fitted to match the given term-structure, by assuming $R(\bm{y}, t)$ is represented by a piecewise-constant function $R_{n+1}(\bm{y})$ over each 
time interval $[t_n, t_{n+1})$. Recall from \eqref{E:discount-factor-node}, this implies that the discount factor $D_n^{\bm{s}}$ can be denoted with explicit dependence on $R_{n+1}$: 
\begin{eqnarray}
    D_n^{\bm{s}}(R_{n+1})
\end{eqnarray}

Before proceeding to the fitting process, let's first introduce the following variables:
\begin{itemize}
    \item $\mathcal{P}_n$: the discount factor, or the price of a zero-coupon bond maturing at time $t_n$ seen at time $t_0$. This is derived from the term-structure.
    \item $P_n^{\bm{m}}$: the Arrow-Debreu price of the payoff function $F_n^{\bm{m}}$ This is obtained by the backward induction \eqref{E:backward-induction} with the discount factor {\em included}:
    \begin{equation}
        P_n^{\bm{m}} = \valuebackward[F_n^{\bm{m}}, t_n].
    \end{equation}
\end{itemize}

The fitting process is a forward stepping process through the following recursive formula: 
\begin{equation}
    \mathcal{P}_{n+1} =  \sum_{\bm{s} \in \indexset_n}\sum_{\bm{e} \in \indexset_{n+1}} D_n^{\bm{s}}(R_{n+1}) \times q_{n}^{\bm{s},\bm{e}} \times P_n^{\bm{s}} \label{E:term-structure-forward-calib}
\end{equation}

For $n=0$, we trivially set $P_0^{\bm{s}} = 1$ for all $\bm{s} \in \indexset_0$,\footnote{It consists of a single node} and match $\mathcal{P}_0 = 0$. In this step, $R_0$ is not applicable.

Suppose that we have fitted up to time $t_n$, keeping the Arrow-Debreu prices $P_n^{\bm{s}}$ for all $\bm{s} \in \indexset_n$. Then,
\begin{enumerate}
    \item solve for $R_{n+1}$ using \eqref{E:term-structure-forward-calib}. 
    \item for each $\bm{e} \in \indexset_{n+1}$, calculate the Arrow-Debreu price $P_{n+1}^{\bm{e}}$:
    \begin{equation}
        P_{n+1}^{\bm{e}} = \sum_{\bm{s} \in \indexset_n} D_n^{\bm{s}}(R_{n+1}) \times q_{n}^{\bm{s},\bm{e}} \times P_n^{\bm{s}}
    \end{equation}
\end{enumerate}

\subsection{Calibration}

The model parameters $\bm{\mu}$ and $\bm{\sigma}$ are calibrated to a set of benchmark derivative contracts by minimizing the aggregate error between the benchmark prices and the lattice-based prices.
Specifically, the calibration process is carried out in the following steps:
\begin{enumerate}
    \item Guess the model parameters $\bm{\mu}$ and $\bm{\sigma}$.
    \item Construct the lattice structure as described in \S~\ref{S:lattice-structure-construction}.
    \item Fit the short rate model to the given term-structure as described in \S~\ref{S:fitting-term-structure}.
    \item For each benchmark derivative contract, calculate the lattice-based price using the backward induction \eqref{E:backward-induction}.
    \item Minimize the aggregate error between the benchmark prices and the lattice-based prices by adjusting the model parameters $\bm{\mu}$ and $\bm{\sigma}$.
\end{enumerate}


\section{Calibration parameters}

\subsection{Time Grid}
\label{S:time-grid}

The parameters required to construct the time grid \eqref{E:time-grid} include:
\begin{itemize}
    \item pairs of horizon and time grid frequency:
    \begin{equation}
        (h_1, f_1), \cdots, (h_K, f_K)
    \end{equation}
    where $h_k$ is the horizon and $f_k$ is the frequency of the time grid for the $k$-th time interval $[t_{k-1}, t_k)$.

    These parameters are pre-configured by instrument type and payment frequency of the underlying instrument. 



\end{itemize}


The time grid \eqref{E:time-grid} is constructed based on a pre-configured set of parameters



\subsection{Node Grid}
\label{S:node-grid}

\appendix

\section{Arrow-Debreu Price Derivation}
\label{S:arrow-debreu}

\section{Natural Cubic Spline}
\label{S:cubic-spline}

The delta payoff function $F_n^{\bm{m}}$ in \eqref{E:delta-payoff} is a product of two natural cubic splines. In the Yield Book, each of the splines is constructed by solving a tridiagonal linear system on the second derivatives of the spline function. 

Consider a sequence of $n$ points:
\begin{eqnarray}
    (x_0, y_0), \cdots, (x_{n-1}, y_{n-1}) 
\end{eqnarray}
Then, the second derivatives, denoted by $S_i$ at the $i$-th point, satisfy the following tridiagonal linear system (see, for example, 
\href{https://charles-oneill.com/projects/cubicspline.pdf}{this paper})
\begin{equation}
    h_{i-1}S_{i-1} + 2(h_{i-1} + h_i)S_i + h_iS_{i+1} = 6\left(\frac{y_{i+1} - y_i}{h_i} - \frac{y_i - y_{i-1}}{h_{i-1}}\right), \quad i = 1, \cdots, n-2
    \label{E:cubic-spline-td-linear-system}
\end{equation}
where $h_i = x_{i+1} - x_{i}$. For the natural cubic spline, we set $S_0 = S_{n-1} = 0$.
Any tridiagonal matrix solver such as the \href{https://en.wikipedia.org/wiki/Tridiagonal_matrix_algorithm}{Thomas algorithm} can be used to solve the system \eqref{E:cubic-spline-td-linear-system}.

Once the second derivatives are obtained, the cubic function $S_i(x)$ over the interval $[x_{i}, x_{i+1})$ is given by
\begin{equation}
    S_i(x) = a_i(x - x_i)^3 + b_i(x - x_i)^2 + c_i(x - x_i) + d_i
\end{equation}
where
\begin{eqnarray}
    a_i & = & \frac{S_{i+1} - S_i}{6h_i} \\
    b_i & = & \frac{S_i}{2} \\
    c_i & = & \frac{y_{i+1} - y_i}{h_i} - \frac{h_i}{6}(S_{i+1} + 2S_i) \\
    d_i & = & y_i
\end{eqnarray}

\section{Gauss-Hermite quadrature}

Since the state variables $y_i(t_{n+1})$ are assumed to be normally distributed, the expected value of a function $f(y_i(t_{n+1}))$ in \eqref{E:green-2} can be computed using the \href{https://en.wikipedia.org/wiki/Gauss%E2%80%93Hermite_quadrature}{Gauss-Hermite quadrature}. The number of quadrature points is a pre-configured model parameter. 

\begin{thebibliography}{}
    
    \bibitem{2fs-citi-mv} 2F-skew model for agency bonds in MRMS and the Yield Book, Level 2 Valuation, Xiaobo Liu, 2005, Citigroup.


\end{thebibliography}


\end{document}